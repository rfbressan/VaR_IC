\documentclass{article}
\usepackage[T1]{fontenc}			% Selecao de codigos de fonte.
\usepackage[utf8]{inputenc}		% Codificacao do documento (conversão automática dos acentos)
\usepackage{longtable}
\usepackage{amsmath,amssymb,mathrsfs, amsthm}	% Comandos matemáticos avançados 
\usepackage{booktabs}
\usepackage{graphicx}	         % Inclusão de gráficos
\usepackage{float}
\usepackage[brazil]{babel}

\begin{document}
	% latex table generated in R 3.4.3 by xtable 1.8-2 package
% Mon Jan  8 15:04:22 2018
\begin{table}[H]
\centering
\caption{Estatísticas descritivas dos retornos (amostra completa de 01/01/2003 a 30/08/2017).} 
\label{tab:descritivas}
\begin{tabular}{lrrrrrr}
  \hline
Descritivas & IBovespa & IPC & IPSA & Merval & S\&P TSE & S\&P500 \\ 
  \hline
Média & 0.00050 & 0.00058 & 0.00045 & 0.00106 & 0.00022 & 0.00028 \\ 
  Mediana & 0.00095 & 0.00093 & 0.00063 & 0.00149 & 0.00074 & 0.00066 \\ 
  Máximo & 0.13677 & 0.10441 & 0.11803 & 0.10432 & 0.09370 & 0.10957 \\ 
  Mínimo & -0.12096 & -0.07266 & -0.07236 & -0.12952 & -0.09788 & -0.09470 \\ 
  Desvp & 0.01739 & 0.01203 & 0.00976 & 0.01981 & 0.01064 & 0.01168 \\ 
  Assimetria & -0.08670 & 0.03784 & -0.01775 & -0.48666 & -0.71699 & -0.33132 \\ 
  Curtose exc. & 4.90756 & 6.58809 & 10.63489 & 3.63347 & 11.84413 & 11.61430 \\ 
  Jarque-Bera & 3655.76824 & 6666.52444 & 17262.83228 & 2125.37846 & 21949.89333 & 20846.77985 \\ 
   & 0.00000 & 0.00000 & 0.00000 & 0.00000 & 0.00000 & 0.00000 \\ 
  Q(10) & 16.27860 & 42.80163 & 111.10837 & 13.32940 & 30.28797 & 59.66372 \\ 
   & 0.00278 & 0.00000 & 0.00000 & 0.01350 & 0.00000 & 0.00000 \\ 
  $Q^2(10)$ & 1299.67247 & 1012.31695 & 919.76012 & 752.94283 & 2384.89328 & 1907.90243 \\ 
   & 0.00000 & 0.00000 & 0.00000 & 0.00000 & 0.00000 & 0.00000 \\ 
  N.obs & 3632.00000 & 3680.00000 & 3658.00000 & 3598.00000 & 3696.00000 & 3692.00000 \\ 
   \hline
\end{tabular}
\end{table}

	
	% latex table generated in R 3.4.3 by xtable 1.8-2 package
% Tue Jan  2 11:49:19 2018
\begin{table}[H]
\centering
\caption{Par\^ametros estimados do modelo eGARCH. Valores p apresentados de acordo 
com erros padrão robustos e valores menores que 0,01 não são mostrados. (Período 
dentro da amostra entre 01/01/2003 a 31/12/2008 ).} 
\label{tab:garchcoef}
\begin{tabular}{lrrrrrr}
  \hline
Parâmetros & IBovespa & IPC & IPSA & Merval & S\&P TSE & S\&P500 \\ 
  \hline
$\mu$ & -0.00104 & -0.00084 & -0.00077 & -0.00079 & -0.00054 & -0.00013 \\ 
   &  &  &  &  &  & (0.41963) \\ 
  $\phi_1$ & -0.00160 & 0.06590 & 0.18028 & -0.00235 & -0.01647 & -0.10160 \\ 
   & (0.93768) &  &  & (0.19700) & (0.64014) &  \\ 
  $\omega$ & -0.30206 & -0.31210 & -0.46252 & -0.72657 & -0.07680 & -0.14485 \\ 
   &  &  &  & (0.48937) &  &  \\ 
  $\alpha_1$ & 0.26221 & 0.19674 & 0.15953 & 0.09108 & 0.13326 & 0.17601 \\ 
   &  &  &  & (0.25260) &  &  \\ 
  $\alpha_2$ & -0.15613 & -0.07566 & -0.08164 & -0.02310 & -0.08771 & -0.07410 \\ 
   &  & (0.04774) & (0.03776) & (0.84697) &  & (0.06315) \\ 
  $\beta_1$ & 0.96259 & 0.96444 & 0.95080 & 0.90939 & 0.99150 & 0.98427 \\ 
   &  &  &  &  &  &  \\ 
  $\gamma_1$ & -0.14167 & 0.05800 & 0.33726 & 0.06958 & 0.09419 & -0.16073 \\ 
   & (0.03516) & (0.40915) &  & (0.68132) & (0.05824) &  \\ 
  $\gamma_2$ & 0.26916 & 0.10179 & -0.02009 & 0.17946 & 0.02824 & 0.27486 \\ 
   &  & (0.19510) & (0.72923) & (0.04203) & (0.60813) &  \\ 
   \hline
\end{tabular}
\end{table}

	
	% latex table generated in R 3.4.2 by xtable 1.8-2 package
% Thu Dec 21 22:04:36 2017
\begin{table}[H]
\centering
\caption{Estatísticas de diagnóstico para o modelo eGARCH. 
               (amostra de trabalho entre 01/01/2003 a 31/12/2008 ).} 
\label{tab:garchstats}
\begin{tabular}{lrrrrrr}
  \hline
Estatística & IBovespa & S\&P500 & S\&P TSE & IPSA & Merval & IPC \\ 
  \hline
Jarque-Bera & 49.43553 & 215.01842 & 140.05923 & 34.73272 & 745.13236 & 110.65485 \\ 
   & 0.00000 & 0.00000 & 0.00000 & 0.00000 & 0.00000 & 0.00000 \\ 
  Q(10) & 5.06865 & 6.62423 & 2.48361 & 4.76927 & 10.79627 & 6.51735 \\ 
   & 0.49583 & 0.29223 & 0.88760 & 0.54183 & 0.04803 & 0.30402 \\ 
  $Q^2(10)$ & 2.06331 & 4.68137 & 4.45027 & 7.97627 & 3.96258 & 2.65357 \\ 
   & 0.93212 & 0.55562 & 0.59236 & 0.17130 & 0.67110 & 0.86672 \\ 
   \hline
\end{tabular}
\end{table}


	% latex table generated in R 3.4.2 by xtable 1.8-2 package
% Thu Dec 21 22:04:36 2017
\begin{table}[H]
\centering
\caption{Parâmetros estimados para o modelo EVT dos resíduos padronizados. 
               (amostra de trabalho entre 01/01/2003 a 31/12/2008 ).} 
\label{tab:evtcoef}
\begin{tabular}{lrrrrrr}
  \hline
 & IBovespa & S\&P500 & S\&P TSE & IPSA & Merval & IPC \\ 
  \hline
Obs. dentro amostra & 1487.00000 & 1511.00000 & 1522.00000 & 1498.00000 & 1495.00000 & 1514.00000 \\ 
  Limiar & 1.67111 & 1.79449 & 1.79155 & 1.69372 & 1.67380 & 1.72553 \\ 
  Número de excessos & 75.00000 & 76.00000 & 77.00000 & 75.00000 & 75.00000 & 76.00000 \\ 
  Parâmetro forma GPD & -0.02626 & 0.17781 & 0.04543 & 0.11917 & 0.11235 & 0.02486 \\ 
  Erro padrão & 0.09054 & 0.13671 & 0.12418 & 0.13746 & 0.10910 & 0.10618 \\ 
  Parâmetro escala GPD & 0.57254 & 0.46220 & 0.56110 & 0.45594 & 0.62512 & 0.57423 \\ 
  Erro padrão & 0.08399 & 0.08195 & 0.09453 & 0.08161 & 0.09897 & 0.08974 \\ 
  Quantil 97.5\% & 2.06927 & 2.13855 & 2.19347 & 2.02384 & 2.12667 & 2.12932 \\ 
  Quantil 99.0\% & 2.57816 & 2.65939 & 2.73554 & 2.50337 & 2.77909 & 2.67082 \\ 
   \hline
\end{tabular}
\end{table}


	% latex table generated in R 3.4.3 by xtable 1.8-2 package
% Wed Dec 20 16:36:07 2017
\begin{table}[H]
\centering
\caption{Percentual de violações. (fora da amostra, dados entre 02/09/2014 e 31/08/2017} 
\label{tab:varviol}
\begin{tabular}{lrrrrrr}
  \hline
Modelo & IBovespa & IPC & IPSA & Merval & S\&P TSE & S\&P500 \\ 
  \hline
Cobertura = 1\% &  &  &  &  &  &  \\ 
  EVT Condicional & 1.34 & 0.67 & 1.07 & 1.37 & 1.20 & 1.06 \\ 
  Normal Condicional & 1.61 & 1.33 & 1.21 & 2.88 & 2.26 & 2.12 \\ 
  t-Student Condicional & 1.61 & 1.33 & 1.21 & 2.88 & 2.26 & 2.12 \\ 
  RiskMetrics & 1.21 & 1.73 & 1.88 & 2.61 & 3.06 & 2.25 \\ 
  EVT Incond. Filtrada & 0.54 & 0.13 & 0.13 & 1.65 & 0.40 & 0.40 \\ 
  Normal Incondicional & 0.54 & 0.27 & 0.13 & 2.19 & 0.13 & 0.40 \\ 
  t-Student Incondicional & 0.27 & 0.13 & 0.00 & 1.51 & 0.00 & 0.00 \\ 
  Cobertura = 2.5\% &  &  &  &  &  &  \\ 
  EVT Condicional & 2.42 & 1.73 & 2.28 & 3.29 & 2.53 & 2.12 \\ 
  Normal Condicional & 2.82 & 2.67 & 2.55 & 3.84 & 4.26 & 2.65 \\ 
  t-Student Condicional & 2.68 & 2.67 & 2.41 & 3.70 & 4.26 & 2.65 \\ 
  RiskMetrics & 2.82 & 3.73 & 2.68 & 3.70 & 4.79 & 3.58 \\ 
  EVT Incond. Filtrada & 1.61 & 0.40 & 0.27 & 2.74 & 1.20 & 0.93 \\ 
  Normal Incondicional & 1.21 & 0.40 & 0.13 & 3.16 & 0.66 & 0.53 \\ 
  t-Student Incondicional & 1.21 & 0.40 & 0.13 & 2.88 & 0.80 & 0.53 \\ 
   \hline
\end{tabular}
\end{table}

	
	% latex table generated in R 3.4.3 by xtable 1.8-2 package
% Tue Jan  2 14:07:14 2018
\begin{longtable}{llrrrrrr}
\caption{Testes estatísticos para o VaR. Teste incondicional de Kupiec e teste de
             independência por duração de Christoffersen e Pelletier (Período fora da 
             amostra entre 02/01/2009 e 30/08/2017).} \\ 
  \hline
Modelo & Estatística & IBovespa & IPC & IPSA & Merval & S\&P TSE & S\&P500 \\ 
  \hline
Cobertura = 1\% &  &  &  &  &  &  &  \\ 
  cevt & uc.LRstat & 0.92 & 0.02 & 1.27 & 1.58 & 0.07 & 0.07 \\ 
  cevt & uc.LRp & 0.34 & 0.89 & 0.26 & 0.21 & 0.79 & 0.80 \\ 
  cevt & uLL & -135.80 & -112.67 & -140.76 & -139.04 & -122.86 & -121.15 \\ 
  cevt & rLL & -136.29 & -113.69 & -140.90 & -140.21 & -123.04 & -123.11 \\ 
  cevt & LRp & 0.32 & 0.15 & 0.59 & 0.13 & 0.54 & 0.05 \\ 
  cnorm & uc.LRstat & 3.78 & 15.16 & 9.15 & 24.00 & 11.22 & 20.57 \\ 
  cnorm & uc.LRp & 0.05 & 0.00 & 0.00 & 0.00 & 0.00 & 0.00 \\ 
  cnorm & uLL & -156.85 & -203.34 & -183.37 & -220.82 & -191.75 & -219.43 \\ 
  cnorm & rLL & -158.08 & -203.63 & -183.38 & -221.81 & -191.76 & -219.62 \\ 
  cnorm & LRp & 0.12 & 0.44 & 0.90 & 0.16 & 0.89 & 0.53 \\ 
  ct & uc.LRstat & 3.78 & 17.94 & 9.15 & 25.66 & 12.43 & 25.32 \\ 
  ct & uc.LRp & 0.05 & 0.00 & 0.00 & 0.00 & 0.00 & 0.00 \\ 
  ct & uLL & -156.85 & -211.19 & -183.37 & -224.15 & -195.79 & -231.07 \\ 
  ct & rLL & -158.08 & -211.52 & -183.38 & -225.62 & -195.79 & -231.16 \\ 
  ct & LRp & 0.12 & 0.42 & 0.90 & 0.09 & 0.96 & 0.66 \\ 
  riskmetrics & uc.LRstat & 4.56 & 20.91 & 19.54 & 29.10 & 34.26 & 32.22 \\ 
  riskmetrics & uc.LRp & 0.03 & 0.00 & 0.00 & 0.00 & 0.00 & 0.00 \\ 
  riskmetrics & uLL & -161.81 & -219.27 & -212.84 & -231.60 & -249.82 & -245.21 \\ 
  riskmetrics & rLL & -162.33 & -219.31 & -215.30 & -233.18 & -249.82 & -246.26 \\ 
  riskmetrics & LRp & 0.31 & 0.79 & 0.03 & 0.08 & 0.98 & 0.15 \\ 
  uevt & uc.LRstat & 1.53 & 4.07 & 1.61 & 0.72 & 0.80 & 2.18 \\ 
  uevt & uc.LRp & 0.22 & 0.04 & 0.21 & 0.40 & 0.37 & 0.14 \\ 
  uevt & uLL & -88.84 & -72.74 & -84.35 & -127.20 & -118.75 & -135.69 \\ 
  uevt & rLL & -89.44 & -74.34 & -89.54 & -131.34 & -136.62 & -149.94 \\ 
  uevt & LRp & 0.27 & 0.07 & 0.00 & 0.00 & 0.00 & 0.00 \\ 
  unorm & uc.LRstat & 0.59 & 0.66 & 7.85 & 6.82 & 1.67 & 0.77 \\ 
  unorm & uc.LRp & 0.44 & 0.42 & 0.01 & 0.01 & 0.20 & 0.38 \\ 
  unorm & uLL & -97.99 & -94.53 & -54.36 & -167.73 & -125.06 & -120.81 \\ 
  unorm & rLL & -99.23 & -99.40 & -58.32 & -170.09 & -145.48 & -136.71 \\ 
  unorm & LRp & 0.11 & 0.00 & 0.00 & 0.03 & 0.00 & 0.00 \\ 
  ut & uc.LRstat & 9.33 & 9.57 & 13.51 & 0.41 & 6.54 & 5.32 \\ 
  ut & uc.LRp & 0.00 & 0.00 & 0.00 & 0.52 & 0.01 & 0.02 \\ 
  ut & uLL & -52.47 & -52.47 & -33.24 & -122.54 & -58.19 & -59.93 \\ 
  ut & rLL & -52.73 & -52.81 & -41.31 & -126.85 & -63.81 & -69.18 \\ 
  ut & LRp & 0.47 & 0.41 & 0.00 & 0.00 & 0.00 & 0.00 \\ 
  Cobertura = 2.5\% &  &  &  &  &  &  &  \\ 
  cevt & uc.LRstat & 0.01 & 0.00 & 0.02 & 1.33 & 0.36 & 0.24 \\ 
  cevt & uc.LRp & 0.93 & 0.99 & 0.89 & 0.25 & 0.55 & 0.63 \\ 
  cevt & uLL & -245.20 & -249.35 & -253.16 & -271.97 & -234.63 & -238.62 \\ 
  cevt & rLL & -245.40 & -249.62 & -253.17 & -273.38 & -234.81 & -238.75 \\ 
  cevt & LRp & 0.53 & 0.46 & 0.86 & 0.09 & 0.55 & 0.60 \\ 
  cnorm & uc.LRstat & 0.11 & 4.38 & 0.67 & 8.71 & 10.10 & 13.21 \\ 
  cnorm & uc.LRp & 0.74 & 0.04 & 0.41 & 0.00 & 0.00 & 0.00 \\ 
  cnorm & uLL & -256.25 & -306.71 & -271.21 & -321.38 & -337.09 & -350.99 \\ 
  cnorm & rLL & -256.47 & -306.78 & -271.39 & -321.65 & -337.52 & -350.99 \\ 
  cnorm & LRp & 0.51 & 0.71 & 0.55 & 0.46 & 0.35 & 0.92 \\ 
  ct & uc.LRstat & 0.11 & 3.86 & 0.17 & 7.99 & 10.10 & 13.21 \\ 
  ct & uc.LRp & 0.74 & 0.05 & 0.68 & 0.00 & 0.00 & 0.00 \\ 
  ct & uLL & -256.25 & -303.27 & -260.36 & -318.04 & -337.09 & -350.99 \\ 
  ct & rLL & -256.47 & -303.33 & -260.51 & -318.29 & -337.52 & -350.99 \\ 
  ct & LRp & 0.51 & 0.75 & 0.57 & 0.48 & 0.35 & 0.92 \\ 
  riskmetrics & uc.LRstat & 4.17 & 8.79 & 5.60 & 6.63 & 21.18 & 23.10 \\ 
  riskmetrics & uc.LRp & 0.04 & 0.00 & 0.02 & 0.01 & 0.00 & 0.00 \\ 
  riskmetrics & uLL & -298.80 & -329.78 & -311.66 & -311.05 & -376.43 & -381.96 \\ 
  riskmetrics & rLL & -302.66 & -330.56 & -313.45 & -311.55 & -376.56 & -383.21 \\ 
  riskmetrics & LRp & 0.01 & 0.21 & 0.06 & 0.32 & 0.61 & 0.11 \\ 
  uevt & uc.LRstat & 5.90 & 13.12 & 2.95 & 0.00 & 0.05 & 0.39 \\ 
  uevt & uc.LRp & 0.02 & 0.00 & 0.09 & 0.95 & 0.82 & 0.53 \\ 
  uevt & uLL & -174.85 & -146.92 & -193.25 & -240.90 & -242.17 & -213.65 \\ 
  uevt & rLL & -183.13 & -154.07 & -203.52 & -244.37 & -257.21 & -234.97 \\ 
  uevt & LRp & 0.00 & 0.00 & 0.00 & 0.01 & 0.00 & 0.00 \\ 
  unorm & uc.LRstat & 6.69 & 6.24 & 19.86 & 3.27 & 0.36 & 3.19 \\ 
  unorm & uc.LRp & 0.01 & 0.01 & 0.00 & 0.07 & 0.55 & 0.07 \\ 
  unorm & uLL & -171.98 & -173.97 & -127.49 & -284.01 & -214.39 & -188.30 \\ 
  unorm & rLL & -179.03 & -183.48 & -131.98 & -290.96 & -234.81 & -203.91 \\ 
  unorm & LRp & 0.00 & 0.00 & 0.00 & 0.00 & 0.00 & 0.00 \\ 
  ut & uc.LRstat & 9.39 & 7.91 & 15.52 & 2.40 & 0.10 & 5.01 \\ 
  ut & uc.LRp & 0.00 & 0.00 & 0.00 & 0.12 & 0.75 & 0.03 \\ 
  ut & uLL & -159.78 & -165.72 & -141.81 & -275.01 & -220.98 & -175.90 \\ 
  ut & rLL & -166.55 & -175.23 & -145.30 & -283.97 & -242.35 & -191.88 \\ 
  ut & LRp & 0.00 & 0.00 & 0.01 & 0.00 & 0.00 & 0.00 \\ 
   \hline
\hline
\label{tab:vartest}
\end{longtable}

	
	% latex table generated in R 3.4.3 by xtable 1.8-2 package
% Wed Jan 17 17:18:48 2018
\begin{table}[H]
\centering
\caption{Sumário para o número de rejeições das hipóteses nulas de um modelo 
corretamente especificado. De seis índices com dois testes, resulta em um total 
de doze rejeições possíveis. (Período fora da amostra entre 02/01/2009 e 30/08/2017).} 
\label{tab:vartest_suma}
\begin{tabular}{lrr}
  \hline
Modelo & Cobertura 1\% & Cobertura 2.5\% \\ 
  \hline
cevt &   1 &   0 \\ 
  cnorm &   5 &   4 \\ 
  ct &   5 &   4 \\ 
  riskmetrics &   7 &   7 \\ 
  uevt &   5 &   8 \\ 
  unorm &   7 &   9 \\ 
  ut &   9 &  10 \\ 
   \hline
\end{tabular}
\end{table}

	

	\begin{figure}[H]
		\centering
		\includegraphics[width=1\linewidth]{figs/artigo-retornos}
		\caption{Retornos dos índices do estudo. Período completo entre 01/01/2003 a 30/08/2017.}
		\label{fig:artigo-retornos}
	\end{figure}

	\begin{figure}[H]
		\centering
		\includegraphics[width=1\linewidth]{figs/artigo-qqplots}
		\caption{Análise de normalidade dos retornos através de gráficos quantil-quantil.}
		\label{fig:artigo-qqplots}
	\end{figure}

	\begin{figure}[H]
		\centering
		\includegraphics[width=1\linewidth]{figs/artigo-gpdfit}
		\caption{Qualidade do ajuste dos dados de inovações em excesso contra uma GPD de referência. Período dentro da amostra.}
		\label{fig:artigo-gpdfit}
	\end{figure}

	\begin{figure}[H]
		\centering
		\includegraphics[width=1\linewidth]{figs/artigo-ibovevt}
		\caption{Teste fora da amostra para o Ibovespa. O modelo EVT condicional (linha pontilhada) é muito mais \textbf{reativo a mudanças na volatilidade que o modelo EVT incondicional filtrado (linha tracejada)}.}
		\label{fig:artigo-sp500evt}
	\end{figure}

\end{document}