%%%%%%%%%%%%%%%%%%%%%%%%%%%%%%%%%%%%%%%%%%%%%%%%%%%%%%%%%%%%%%%%%%%%%%%%%%%%%%%%%%%%%%%%%%%%%%
% Template Beamer 
% Based on MIT Beamer Template e Senac
% Cores verde e vermelho tentam seguir o padrão visual da Udesc
%%%%%%%%%%%%%%%%%%%%%%%%%%%%%%%%%%%%%%%%%%%%%%%%%%%%%%%%%%%%%%%%%%%%%%%%%%%%%%%%%%%%%%%%%%%%%% 

\usepackage{graphicx,url}
\usepackage[brazil]{babel}   
\usepackage{textpos}

\batchmode
% \usepackage{pgfpages}
% \pgfpagesuselayout{4 on 1}[letterpaper,landscape,border shrink=5mm]
\usepackage{amsmath,amssymb,enumerate,epsfig,bbm,calc,color,ifthen,capt-of}

%-------------------------Declara figura do Logo-----------------------------------------------------
\pgfdeclareimage[height=0.8cm]{Marca_Udesc}{Marca_Udesc.pdf}
%\logo{\pgfuseimage{Marca_Udesc}\vspace*{7.0cm}}

%-------------------------Este código faz o menuzinho bacana na parte superior do slide------------
\AtBeginSection[]
{
%  \begin{frame}<beamer>
%    \frametitle{Sumário}
%    \tableofcontents[currentsection]
%  \end{frame}
}

\AtBeginSubsection{}

% Para ativar os tópicos de forma incremental
%\beamerdefaultoverlayspecification{<+->}
% -----------------------------------------------------------------------------
% -----Página de título sem o logo -----------------------------------
%\renewcommand{\titlepage}{\setbeamertemplate{logo}{}\titlepage}
%\setbeamertemplate{logo}{}\titlepage

% ------------ Logo na parte superior direita --------------------------------
\addtobeamertemplate{frametitle}{}{%
\begin{textblock*}{100mm}(.85\textwidth,-0.9cm)
\pgfuseimage{Marca_Udesc}
\end{textblock*}}
%
% - Para criar duas colunas em algum slide da apresentacao 
% - a partir do Rmarkdown
\def\begincols{\begin{columns}}
\def\begincol{\begin{column}}
\def\endcol{\end{column}}
\def\endcols{\end{columns}}

%%---Gerador de Sumário---------------------------------------------------------
%\section[]{}
%\begin{frame}{Sumário}
%  \tableofcontents
%\end{frame}
%---Fim do Sumário------------------------------------------------------------
